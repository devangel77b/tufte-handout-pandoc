% Options for packages loaded elsewhere
\PassOptionsToPackage{unicode}{hyperref}
\PassOptionsToPackage{hyphens}{url}
\PassOptionsToPackage{svgnames,hyperref}{xcolor}
%
\documentclass{tufte-handout}

% preliminary declarations
\RequirePackage[T1]{fontenc}
\RequirePackage[utf8]{inputenc}
\RequirePackage{textcomp} % provide euro and other symbols
\RequirePackage{microtype}
\RequirePackage{amsmath,amsfonts,amssymb} % use AMS math

% From preamble.tex
\RequirePackage[american]{babel} % for course dates?
\RequirePackage[american,inputamerican]{isodate} % for course dates?
%\dateinputformat{american}
\RequirePackage{amsmath,amsfonts,amssymb} % use AMS math
\RequirePackage[makeroom]{cancel} % for when I cancel things in math
\RequirePackage{graphicx} % use graphics
%\RequirePackage{svg,svg-extract} % check if conflicts with Beamer? 
\RequirePackage{booktabs} % for nice tables
\RequirePackage{siunitx}
\sisetup{per-mode=symbol} % to get e.g. m/s, instead of [m s^{-1}]
\DeclareSIUnit\foot{ft}
\DeclareSIUnit\pound{lb}
\DeclareSIUnit\ounce{oz}
\DeclareSIUnit\inch{in}
\DeclareSIUnit\rpm{rpm}
\DeclareSIUnit\fahrenheit{F}
\DeclareSIUnit\bit{bit}
\DeclareSIUnit\byte{B}
\RequirePackage[version=4]{mhchem} 
% for chemical stuff, replaces chemfig, chemformula for pandoc compatibility 
\RequirePackage{upquote}
\RequirePackage[plain]{fancyref} % \fref, \Fref... 
\RequirePackage{bibentry} 
%\RequirePackage{sidenotes}
\RequirePackage{xurl}
%\RequirePackage{bookmark}

% oddballs to avoid option clashes with tufte
%\RequirePackage{xcolor}

% svg and minted require shell-escape so make them loaded by user
%\RequirePackage{minted} % replaces use of listings, for pandoc compatibility
%\RequirePackage{svg,svg-extract} % use SVG figures




\setlength{\emergencystretch}{3em} % prevent overfull lines
\providecommand{\tightlist}{%
  \setlength{\itemsep}{0pt}\setlength{\parskip}{0pt}}

\hypersetup{
  pdftitle={Kinematics and forces},
  pdfauthor={Dennis Evangelista},
  hidelinks,
  pdfcreator={LaTeX via pandoc}}

\title{Kinematics and forces}
\author{Dennis Evangelista}
\date{\today}

\begin{document}
\maketitle

\hypertarget{kinematics}{%
\section{Kinematics}\label{kinematics}}

Kinematics is the quantitative study of motion. To describe how an
object moves, we typically describe its position, velocity, and
acceleration in some useful coordinate system / coordinate frame.

\hypertarget{position-velocity-acceleration}{%
\subsection{Position, velocity,
acceleration}\label{position-velocity-acceleration}}

\textbf{Position} is a
\textbf{vector}\sidenote{A vector is a number and a direction. To convince yourself direction matters, imagine flying up from the ground \qty{33}{\meter}, versus flying down into the ground \qty{33}{\meter}; are those different? Would the direction matter? Yes!

} and describes where an object is in space relative to an established
coordinate system. Typical SI units for position are \unit{\meter}. I
normally use \(\vec{r}\), \(x\), \(y\), or \(z\) as variables to
describe position, often with an arrow over them to remind myself they
are vectors.

\textbf{Velocity} is also a vector and describes the \textbf{time rate
of change of position}. Its units are \unit{\meter\per\second}. I
normally use \(\vec{v}\) to represent velocity. Considering \(\Delta\)
or \(d\) as a ``change in'', velocity becomes \begin{align}
\text{velocity}, [\si{\meter\per\second}] &= \dfrac{\text{change in position}}{\text{change in time}} \\
&= \dfrac{\Delta \vec{x}}{\Delta t} \\
&= \dfrac{d\vec{x}}{dt} \\
\end{align}
\sidenote{The last form, read "dee x dee tee," is how velocity is typically written as a "derivative" in calculus, an advanced type of math that was invented partially to make physics easier to understand. You are not responsible for this in Physics 9.

} The ``time rate of change of position'' relationship means that
\textbf{velocity is like the slope of a position versus time graph,} and
that \textbf{position is like the area under a velocity versus time
graph.}

\textbf{Acceleration} is obtained from doing a Madlibs sort of
thing\ldots{} we take it as the \textbf{time rate of change of
velocity}. Its units are \unit{\meter\per\second\squared}. I normally
use \(\vec{a}\) to represent acceleration.

\begin{align}
\text{acceleration}, [\si{\meter\per\second\squared}] &= \dfrac{\text{change in velociy}}{\text{change in time}} \\
&= \dfrac{\Delta \vec{v}}{\Delta t} \\
&= \dfrac{d \vec{v}}{dt}
\end{align}

The ``time rate of change of velocity'' relationship means that
\textbf{acceleration is like the
slope\sidenote{Remember slope is rise over run in $y=mx+b$ in math; here compare to $x=vt+x_0$ and $v=at+v_0$.} of a velocity versus time graph,} and that \textbf{velocity is like
the area under an acceleration versus time graph.}

\hypertarget{related-scalar-quantities}{%
\subsection{Related scalar quantities}\label{related-scalar-quantities}}

Vector position and \textbf{displacement} are related to the scalar
quantity \textbf{distance}. \textbf{Distance} will normally be
considered to be the total distance traveled along a path from
\(\vec{x}_0\) to \(\vec{x}_f\) and is only a simple number (e.g.\ 50
miles) so it is a
\textbf{scalar}\sidenote{A scalar is just a plain old number; distance, temperature, mass don’t really have a direction to them so they are not vectors.

} quantity. \textbf{Displacement} (\(\vec{d}\)) is a vector quantity and
is normally taken as \(\vec{x}_f - \vec{x}_0\).

Vector velocity is related to the scalar quantity \textbf{speed}. Speed
is normally either the scalar magnitude of velocity, e.g.~instantaneous
speed, \(|v|\), or the total distance traveled divided by the total time
(average speed).

\hypertarget{d-motion-with-constant-velocity}{%
\subsection{1D motion with constant
velocity}\label{d-motion-with-constant-velocity}}

The following equations hold for 1D motion at \textbf{constant (linear)
velocity}, which means the speed \emph{and
direction}\sidenote{In Physics 9 you are not responsible for circular motion, in which speed might be constant but direction is constantly changing, nor are you responsible for angular velocity

} of the object are not changing:

\begin{align}
x(t) &= v t + x_0 \\
v(t) &= v\ \text{(constant)} \\
a(t) &= 0
\end{align}

Examples of 1D motion at constant velocity would include things like a
skier moving at \qty{5}{\meter\per\second} north; a softball in space
with no forces acting on it; or an object that is not accelerating. The
big example fo this is when we pushed people on chairs at constant
speed, and also the horizontal component of the marble shooting
experiment.

\hypertarget{d-motion-with-constant-acceleration}{%
\subsection{1D motion with constant
acceleration}\label{d-motion-with-constant-acceleration}}

The following equations hold for 1D motion with \textbf{constant
(linear) acceleration}, which means there is a net force acting on the
object that makes it go faster or slower.

\begin{align}
x(t) &= \dfrac{1}{2} a t^2 + v_0 t + x_0 \\
v(t) &= at + v_0 \\
a(t) &= a\ \text{(constant)} 
\end{align}

Examples of 1D motion with constant acceleration include the case of a
ball dropped from the second floor balcony, or a car at a stop light
when it hits the accelerator and before shifting gears, or a rocket ship
firing a thruster with a specified force output. The big example of this
is when we dropped stuff from the balcony; as well as the vertical
component of the marble shooting experiment.

\textbf{For this test, do not expect 3D motion or motions that cannot be
modeled as either constant velocity or constant acceleration.}

\hypertarget{alternate-form-of-equations}{%
\subsection{Alternate form of
equations}\label{alternate-form-of-equations}}

When working with
displacement\sidenote{$d$ here means displacement and does not mean "change in"

} \(d\), alternate forms of the equations of motion for constant
accelerations
are\sidenote{See The Physics Classroom, https://www.physicsclassroom.com

}

\begin{align}
d &= v_i t + \dfrac{1}{2} a t^2 \\
v_f^2 &= v_i^2 + 2 a d \\
v_f &= v_i + a t \\
d &= \dfrac{v_i+v_f}{2} t \\
\end{align}

These have the advantage of giving displacement in terms of the initial
and final velocities (\(v_i\) and \(v_f\), respectively), the
acceleration \(a\) and the time \(t\). All are equivalent so use
whatever form you are most comfortable with. CrossRefHere2 can be
derived from energy conservation and provides a handy way to find final
velocity when displacement is known. CrossRefHere3 has a term related to
the average speed \(\dfrac{v_i+v_f}{2}\) during the time interval \(t\)
(sometimes given as \(\Delta t\)); which may be useful when considering
what quantities a problem wishes you to compute.

\hypertarget{forces}{%
\section{Forces}\label{forces}}

\textbf{Forces} can come from things like weight
(gravity)\sidenote{The force of gravity on an object of mass m is $F = mg$ directed downwards, where $g=\qty{9.8}{\meter\per\second\squared}$ is the acceleration of gravity at the surface of the Earth. You can also get this from Newton’s second law. $g$ will be provided for you in the equation sheet.

}, aerodynamic lift or drag, thrust from an engine, friction from the
ground, normal forces from the ground; forces also arise from charges,
electric and magnetic fields, reaction forces, etc etc. The SI unit of
force is a newton, \unit{\newton}, defined as
\(\qty{1}{\newton}=\qty{1}{\kilo\gram\meter\per\second\squared}\). You
may also see force specified as pounds force (lbf) when working with
non-SI units (such as in specifying the thrust of a jet engine). The
form of Newton's second law used in Physics 9 is

\begin{align}
\text{net force} &= \text{mass} \cdot \text{acceleration} \\
\sum\vec{F} &= m\vec{a}
\end{align}

\hypertarget{free-body-diagrams}{%
\subsection{Free body diagrams}\label{free-body-diagrams}}

\textbf{Free body diagrams} are ways to visualize the force acting on an
object. The object is drawn isolated from the rest of the universe, and
arrows are used to show the forces that are acting on it and their
location and directions. This is a useful tool in analyzing the
mechanics of all sorts of things, but \textbf{for this test you may only
be asked to draw free body diagrams of very simple
situations}\sidenote{For example, falling objects with or without air drag; rockets sitting on a launch pad, etc.

}.

\hypertarget{newtons-laws}{%
\subsection{Newton's laws}\label{newtons-laws}}

\begin{enumerate}
\def\labelenumi{\arabic{enumi}.}
\tightlist
\item
  An object at rest will stay at rest, unless acted upon by an outside
  force; an object in motion will stay in motion unless acted upon by an
  outside force.
\item
  If there is an outside force acting, the sum of the forces will equal
  the time rate of change of
  momentum\sidenote{Momentum is $\vec{p} = m \vec{v}$ and is the product of mass and velocity. Since in Physics 9 we are usually dealing with objects of constant mass, a simpler version of this law is just $\sum\vec{F} = m \vec{a}$, where $\sum\vec{F}$ is the net force or the sum of the forces, $m$ is mass, and $\vec{a}$ is acceleration.
  }
\item
  For every action, there is an equal and opposite reaction.
\end{enumerate}

The first law could also be viewed as the case of the second law where
\(\vec{a}=0\), and (together with the third law) is studied most in
\textbf{statics}\sidenote{First course taken by many engineers in college…

}, when considering the balance of forces for and within objects that
are not accelerating.

The second law shows up most in cases where objects are accelerating,
such as in studies of vehicles, maneuvers, propulsion systems, or
generally in
\textbf{dynamics}\sidenote{Second course taken by many engineers in college…

}.

The third law does not show up very often; one place in engineering is
when an object pushing on the ground or some other object, the force it
feels is often called the \textbf{reaction force} and given the symbol
\(R\) in reference to the third law.
\end{document}